\documentclass[12pt,a4paper]{article}

% Paquetes necesarios
\usepackage[utf8]{inputenc}
\usepackage[spanish]{babel}
\usepackage[T1]{fontenc}
\usepackage{geometry}
\usepackage{fancyhdr}
\usepackage{titlesec}
\usepackage{enumitem}
\usepackage{amsmath}
\usepackage{amsfonts}
\usepackage{amssymb}
\usepackage{graphicx}
\usepackage{url}
\usepackage{hyperref}
\usepackage{xcolor}

% Configuración de márgenes
\geometry{
    left=2.5cm,
    right=2.5cm,
    top=3cm,
    bottom=3cm
}

% Configuración de encabezados y pie de página
\pagestyle{fancy}
\fancyhf{}
\fancyhead[L]{Anteproyecto TFM}
\fancyhead[R]{Máster en Big Data y Business Analytics}
\fancyfoot[C]{\thepage}

% Configuración de hipervínculos
\hypersetup{
    colorlinks=true,
    linkcolor=black,
    urlcolor=blue,
    citecolor=black
}

% Título del documento
\title{\textbf{Anteproyecto de Trabajo Fin de Máster (TFM)}}
\author{}
\date{}

\begin{document}

\maketitle
\thispagestyle{empty}

% Información adicional de portada debajo del título
\begin{center}
\textbf{Máster en Big Data y Business Analytics} \\
Autor: Juan Carlos Garzon \\
Tutor: Juan Manuel Moreno Lamparero
\end{center}

\vspace{1cm}

\section{Título del Proyecto}

Análisis de Datos y Procesamiento de Lenguaje Natural para la Extracción de Opiniones y Modelado de Tópicos en Restaurantes: Un Enfoque de Big Data y Ciencia de Datos Aplicado al Estudio Integral del Sector Gastronómico

\section{Breve Descripción}

Este proyecto tiene como propósito desarrollar una solución integral de análisis de datos que transforme las reseñas de restaurantes no estructuradas en insights valiosos para la toma de decisiones. Utilizando el dataset público de Yelp, se aplicarán técnicas avanzadas de Procesamiento de Lenguaje Natural (NLP) para extraer automáticamente el sentimiento y los temas principales de miles de reseñas de restaurantes.

La solución incluye un pipeline completo de datos: desde la ingesta y almacenamiento en MongoDB, pasando por el procesamiento con modelos de inteligencia artificial, hasta la presentación de resultados a través de una aplicación web interactiva desarrollada en Streamlit. El resultado final será un cuadro de mandos (dashboard) que permita a propietarios de restaurantes y consumidores explorar patrones de opinión, tendencias de satisfacción y temas específicos de discusión de manera visual e intuitiva.

\section{Objetivos}

\subsection{Objetivo General}

Desarrollar una solución de datos de extremo a extremo que ingiera, procese, analice y visualice reseñas de restaurantes, culminando en una aplicación web interactiva que sirva como herramienta analítica y cuadro de mandos.

\subsection{Objetivos Específicos}

\begin{enumerate}[label=\arabic*.]
    \item Ingestar y estructurar los datos del Yelp Open Dataset en una base de datos MongoDB para una gestión eficiente de datos semiestructurados.
    
    \item Realizar un análisis exploratorio de los datos para identificar patrones iniciales en calificaciones, popularidad y características de los restaurantes.
    
    \item Implementar un modelo de análisis de sentimiento utilizando técnicas de inteligencia artificial (modelos pre-entrenados de Transformers) para clasificar automáticamente la polaridad de cada reseña.
    
    \item Aplicar técnicas de modelado de tópicos (LDA) para descubrir y cuantificar los temas latentes en el corpus de reseñas mediante aprendizaje no supervisado.
    
    \item Desarrollar un cuadro de mandos interactivo utilizando Streamlit y Plotly que permita la exploración dinámica de sentimientos, tópicos y métricas de restaurantes.
\end{enumerate}

\section{Datos que se van a Utilizar}

Para este proyecto se utilizará el \textbf{Yelp Open Dataset}, un conjunto de datos público y masivo que contiene información real de negocios, reseñas y usuarios de la plataforma Yelp.

\subsection{Características del Dataset}

\begin{itemize}[label=\textbullet]
    \item \textbf{Fuente:} Yelp Open Dataset (\url{https://business.yelp.com/data/resources/open-dataset/})
    
    % Imagen de la web del dataset
    \begin{figure}[h!]
        \centering
        \includegraphics[width=0.7\textwidth]{images/fig1}
        \caption{Página web oficial del Yelp Open Dataset}
    \end{figure}
    \item \textbf{Formato:} Archivos JSON con datos semiestructurados
    \item \textbf{Volumen:} Más de 6 millones de reseñas y 150,000 negocios
    \item \textbf{Geografía:} Múltiples ciudades en Estados Unidos
    \item \textbf{Contenido:} Reseñas textuales, calificaciones numéricas, metadatos de negocios
    \item \textbf{Licencia:} Disponible para uso académico y de investigación
\end{itemize}

\subsection{Estructura de los Datos}

El dataset incluye cinco archivos principales:
\begin{enumerate}
    \item \texttt{yelp\_academic\_dataset\_business.json} - Información de negocios (ubicación, categorías, atributos)
    \item \texttt{yelp\_academic\_dataset\_review.json} - Reseñas de usuarios con texto y calificaciones
    \item \texttt{yelp\_academic\_dataset\_user.json} - Información de usuarios
    \item \texttt{yelp\_academic\_dataset\_checkin.json} - Check-ins en negocios
    \item \texttt{yelp\_academic\_dataset\_tip.json} - Tips cortos de usuarios
\end{enumerate}

\section{Resultados Esperados con Énfasis en Streamlit}

\subsection{Aplicación Web Interactiva (Producto Principal)}

\begin{itemize}[label=\textbullet]
    \item \textbf{Cuadro de Mandos Principal:} Una aplicación Streamlit que sirva como dashboard ejecutivo para la exploración de datos de restaurantes.
    
    \item \textbf{Funcionalidades Interactivas:} La aplicación permitirá la exploración y visualización flexible de los datos, incluyendo filtros, gráficos y herramientas interactivas adaptadas a la información disponible.
    
    \item \textbf{Componentes del Dashboard:} El cuadro de mandos integrará diferentes visualizaciones y métricas relevantes, ajustándose a los resultados obtenidos y a la naturaleza de los datos procesados.
\end{itemize}

\subsection{Entregables Técnicos}

\begin{itemize}[label=\textbullet]
    \item Repositorio GitHub con código documentado y reproducible
    \item Base de datos MongoDB con los datos procesados
    \item Trabajo de Fin de Máster (TFM)
\end{itemize}

\section{Módulos del Máster Integrados}

Este proyecto integra competencias clave de varios módulos del Máster en Big Data y Business Analytics. A continuación, se detalla cómo se aplican en el desarrollo del trabajo:

\begin{itemize}[label=\textbullet]
    \item \textbf{M1. Fundamentos de tratamiento de datos para Data Science:} Permite la preparación, análisis y transformación de los datos textuales del proyecto, aplicando Python, Pandas y NumPy, así como lógica de scripts y procesamiento de datos en terminal.
    
    \item \textbf{M2. Business Intelligence:} Proporciona el marco estratégico para transformar los datos en conocimiento útil para el negocio. Se emplean conceptos de BI como la interpretación de insights, la creación de dashboards y la generación de valor para los actores del sector gastronómico.
    
    \item \textbf{M4. Minería de Texto y Procesamiento del Lenguaje Natural (PLN):} Es el núcleo metodológico del proyecto, aplicando técnicas de análisis de sentimiento, clasificación y modelado de tópicos con herramientas como NLTK y modelos de Transformers (BERT, RoBERTa) para extraer conocimiento de grandes volúmenes de reseñas no estructuradas.
    
    \item \textbf{M5. Inteligencia de Negocio y Visualización:} Se aplica en la creación de un cuadro de mandos visual e interactivo mediante Streamlit y Plotly, empleando buenas prácticas de visualización para que los resultados sean comprensibles y útiles para la toma de decisiones.
    
    \item \textbf{M7. Almacenamiento e Integración de Datos:} Aporta los fundamentos sobre bases de datos NoSQL, especialmente el uso de MongoDB para almacenar y consultar datos semiestructurados en formato JSON, aplicando modelado documental y consultas agregadas para gestionar eficazmente la información.
\end{itemize}

\section{Cuadro de Mandos - Especificaciones Técnicas}

\subsection{Arquitectura del Dashboard}

\begin{itemize}[label=\textbullet]
    \item \textbf{Framework Principal:} Streamlit para desarrollo rápido de aplicaciones web de datos
    \item \textbf{Visualizaciones:} Plotly Express y Plotly Graph Objects para gráficos interactivos
    \item \textbf{Mapas:} Integración con Folium/Plotly para visualización geoespacial
\end{itemize}

\subsection{Funcionalidades del Cuadro de Mandos}

\begin{enumerate}[label=\arabic*.]
    \item \textbf{Vista Ejecutiva:} KPIs principales y métricas de alto nivel
    \item \textbf{Análisis de Sentimientos:} Distribución y tendencias de opiniones por restaurante
    \item \textbf{Explorador de Tópicos:} Análisis temático interactivo de reseñas
    \item \textbf{Insights Geográficos:} Mapas de calor de satisfacción por zonas geográficas
\end{enumerate}

\subsection{Tecnologías de Implementación}

\begin{itemize}[label=\textbullet]
    \item \textbf{Desarrollo:} Python 3.11, Streamlit, Plotly, Pandas
    \item \textbf{Base de Datos:} MongoDB (NoSQL)
    \item \textbf{Control de Versiones:} Git/GitHub
    \item \textbf{Procesamiento NLP:} Transformers (Hugging Face), scikit-learn, NLTK, spaCy
\end{itemize}

\end{document} 